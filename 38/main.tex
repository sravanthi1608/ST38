%\iffalse
\let\negmedspace\undefined
\let\negthickspace\undefined
\documentclass[journal,12pt,twocolumn]{IEEEtran}
\usepackage{cite}
\usepackage{amsmath,amssymb,amsfonts,amsthm}
\usepackage{algorithmic}
\usepackage{graphicx}
\usepackage{textcomp}
\usepackage{xcolor}
\usepackage{txfonts}
\usepackage{listings}
\usepackage{enumitem}
\usepackage{mathtools}
\usepackage{gensymb}
\usepackage[breaklinks=true]{hyperref}
\usepackage{tkz-euclide} % loads  TikZ and tkz-base
\usepackage{listings}



\newtheorem{theorem}{Theorem}[section]
\newtheorem{problem}{Problem}
\newtheorem{proposition}{Proposition}[section]
\newtheorem{lemma}{Lemma}[section]
\newtheorem{corollary}[theorem]{Corollary}
\newtheorem{example}{Example}[section]
\newtheorem{definition}[problem]{Definition}
%\newtheorem{thm}{Theorem}[section] 
%\newtheorem{defn}[thm]{Definition}
%\newtheorem{algorithm}{Algorithm}[section]
%\newtheorem{cor}{Corollary}
\newcommand{\BEQA}{\begin{eqnarray}}
\newcommand{\EEQA}{\end{eqnarray}}
\newcommand{\define}{\stackrel{\triangle}{=}}
\theoremstyle{remark}
\newtheorem{rem}{Remark}
%\bibliographystyle{ieeetr}
\begin{document}
%
\providecommand{\pr}[1]{\ensuremath{\Pr\left(#1\right)}}
\providecommand{\prt}[2]{\ensuremath{p_{#1}^{\left(#2\right)} }}        % own macro for this question
\providecommand{\qfunc}[1]{\ensuremath{Q\left(#1\right)}}
\providecommand{\sbrak}[1]{\ensuremath{{}\left[#1\right]}}
\providecommand{\lsbrak}[1]{\ensuremath{{}\left[#1\right.}}
\providecommand{\rsbrak}[1]{\ensuremath{{}\left.#1\right]}}
\providecommand{\brak}[1]{\ensuremath{\left(#1\right)}}
\providecommand{\lbrak}[1]{\ensuremath{\left(#1\right.}}
\providecommand{\rbrak}[1]{\ensuremath{\left.#1\right)}}
\providecommand{\cbrak}[1]{\ensuremath{\left\{#1\right\}}}
\providecommand{\lcbrak}[1]{\ensuremath{\left\{#1\right.}}
\providecommand{\rcbrak}[1]{\ensuremath{\left.#1\right\}}}
\newcommand{\sgn}{\mathop{\mathrm{sgn}}}
\providecommand{\abs}[1]{\left\vert#1\right\vert}
\providecommand{\res}[1]{\Res\displaylimits_{#1}} 
\providecommand{\norm}[1]{\left\lVert#1\right\rVert}
%\providecommand{\norm}[1]{\lVert#1\rVert}
\providecommand{\mtx}[1]{\mathbf{#1}}
\providecommand{\mean}[1]{E\left[ #1 \right]}
\providecommand{\cond}[2]{#1\middle|#2}
\providecommand{\fourier}{\overset{\mathcal{F}}{ \rightleftharpoons}}
\newenvironment{amatrix}[1]{%
  \left(\begin{array}{@{}*{#1}{c}|c@{}}
}{%
  \end{array}\right)
}
%\providecommand{\hilbert}{\overset{\mathcal{H}}{ \rightleftharpoons}}
%\providecommand{\system}{\overset{\mathcal{H}}{ \longleftrightarrow}}
	%\newcommand{\solution}[2]{\textbf{Solution:}{#1}}
\newcommand{\solution}{\noindent \textbf{Solution: }}
\newcommand{\cosec}{\,\text{cosec}\,}
\providecommand{\dec}[2]{\ensuremath{\overset{#1}{\underset{#2}{\gtrless}}}}
\newcommand{\myvec}[1]{\ensuremath{\begin{pmatrix}#1\end{pmatrix}}}
\newcommand{\mydet}[1]{\ensuremath{\begin{vmatrix}#1\end{vmatrix}}}
\newcommand{\myaugvec}[2]{\ensuremath{\begin{amatrix}{#1}#2\end{amatrix}}}
\providecommand{\rank}{\text{rank}}
\providecommand{\pr}[1]{\ensuremath{\Pr\left(#1\right)}}
\providecommand{\qfunc}[1]{\ensuremath{Q\left(#1\right)}}
	\newcommand*{\permcomb}[4][0mu]{{{}^{#3}\mkern#1#2_{#4}}}
\newcommand*{\perm}[1][-3mu]{\permcomb[#1]{P}}
\newcommand*{\comb}[1][-1mu]{\permcomb[#1]{C}}
\providecommand{\qfunc}[1]{\ensuremath{Q\left(#1\right)}}
\providecommand{\gauss}[2]{\mathcal{N}\ensuremath{\left(#1,#2\right)}}
\providecommand{\diff}[2]{\ensuremath{\frac{d{#1}}{d{#2}}}}
\providecommand{\myceil}[1]{\left \lceil #1 \right \rceil }
\newcommand\figref{Fig.~\ref}
\newcommand\tabref{Table~\ref}
\newcommand{\sinc}{\,\text{sinc}\,}
\newcommand{\rect}{\,\text{rect}\,}
%%
%	%\newcommand{\solution}[2]{\textbf{Solution:}{#1}}
%\newcommand{\solution}{\noindent \textbf{Solution: }}
%\newcommand{\cosec}{\,\text{cosec}\,}
%\numberwithin{equation}{section}
%\numberwithin{equation}{subsection}
%\numberwithin{problem}{section}
%\numberwithin{definition}{section}
%\makeatletter
%\@addtoreset{figure}{problem}
%\makeatother

%\let\StandardTheFigure\thefigure
\let\vec\mathbf

\bibliographystyle{IEEEtran}


\vspace{3cm}



\bigskip

\renewcommand{\thefigure}{\theenumi}
\renewcommand{\thetable}{\theenumi}
%\renewcommand{\theequation}{\theenumi}
Question : Let ${N(t)}_{t\ge 0}$ be a Poisson process with rate 1. Consider the following statements. 
\begin{enumerate}[label=(\alph*)]
\item $pr\brak{N(3)=3|N(5)=5}=\comb{5}{3}\left(\frac{3}{5}\right)^3 \left(\frac{2}{5}\right)^2$
\item If $S_5$ denotes the time of occurrence of the $5^{th}$ event for the above Poisson process,then $E(S_5|N(5)=3)=7$ \\
\end{enumerate}
Which of the above statements is/are true?\\
\begin{enumerate}[label=(\roman*)]
\item only (a)
\item only (b)
\item Both (a) and (b)
\item Neither (a) and (b)
\end{enumerate}
\solution \\
\begin{enumerate}[label=(\alph*)]
\item Using the Poisson probability formula,\\
 \begin{align}
 pr\brak{N(t)=k} = \frac{(\lambda t)^k e^{-\lambda t}}{k!} 
  \end{align}
 here $\lambda$ is 1 
 \begin{align}
 pr\brak{N(t)=k} = \frac{(t)^k e^{-t}}{k!}\\
 pr\brak{N(3)=3} = \frac{(3)^3 e^{-3}}{3!}\\
 pr\brak{N(5)=5} = \frac{(5)^5 e^{-5}}{5!}
 \end{align}
  $X=Y+Z$\\
 Conditional pmf Y given ${X=x_o}$,\\
 $X$ and $Y$ are equivalent events\\
  \begin{align}
 pr\brak{Y=y|X=x_o}&=\frac{pr\brak{Y=y,X=x_o}}{pr\brak{X=x_o}}\\
 &=\frac{pr\brak{Y=y,Z=x_o-y}}{pr\brak{X=x_o}} 
  \end{align}
  $Y$ and $Z$ are independent,
  \begin{align}
  pr\brak{Y=y,Z=x_o-y}=pr\brak{Y=y}pr\brak{Z=x_o-y}
  \end{align}
  $X,Y $and $Z$ in poissions distribution,
  \begin{align}
  pr\brak{Y=y}=\frac{(\lambda)^y e^{-\lambda}}{y!}\\
  pr\brak{X=x_o}=\frac{(\lambda)^{x_o} e^{-\lambda}}{x_o!}\\
  pr\brak{Z=x_o-y}=\frac{(\lambda)^{x_o-y} e^{-\lambda}}{(x_o-y)!}
  \end{align}
 From conditional probability,
  \begin{align}
 pr\brak{N(3)=3|N(5)=5}&=\frac{\frac{(3)^3 e^{-3}}{3!}\frac{(2)^2 e^{-2}}{2!}}{\frac{(5)^5 e^{-5}}{5!}}\\
 &=\frac{(3)^3(2)^2}{(5)^5}\frac{5!}{3!2!}\\
 &=\comb{5}{3}\left(\frac{3}{5}\right)^3 \left(\frac{2}{5}\right)^2
 \end{align}
 Hence statement (a) is true.
\item The expected value $E(S_n)$ of the time at which the $n_{th}$ event occurs in a Poisson process with rate $\lambda$ is\\
\begin{align}
E(S_2)=\frac{n}{\lambda}
\end{align}
The conditional expectation $E(S_n|N(t)=x)$ represents the expected time at which the $n^{th}$ event occurs given that exactly $x$ events have occurred in the first $t$ units of time in a Poisson process with rate $\lambda$ is given by.\\
\begin{align}
E(S_n|N(t)=x)&=t+E(S_{n-x})
\end{align}
By the law of total expectation,
\begin{align}
E(S_n | N(t) = x) &= E(E(S_n | N(t) = x, N(t))) \\
&= E(E(S_x + S_{n-x} | N(t) = x, N(t) = x))\\
&= E(E(S_x + S_{n-x} | N(t) = x))\\
&=E(t + E(S_{n-x})) \\
&=E(t) + E(E(S_{n-x}))\\
 &= t + E(S_{n-x})
 \end{align}
 From above result,
 \begin{align}
 E(S_5|N(5)=3)&=5+E(S_2)\\
 &=5+2\\
 &=7
 \end{align}
 Hence statement (b) is true.
\end{enumerate}
Both (a) and (b) are true.
\end{document}
